\documentclass{article}
\usepackage[T1]{fontenc}
\usepackage{xparse}
\usepackage{enumitem}
\setlist[description]{
  font={\sffamily\bfseries},
  labelsep=0pt,
  labelwidth=\transcriptlen,
  leftmargin=\transcriptlen,
}

\newlength{\transcriptlen}

\NewDocumentCommand {\setspeaker} { mo } {%
  \IfNoValueTF{#2}
  {\expandafter\newcommand\csname#1\endcsname{\item[#1:]}}%
  {\expandafter\newcommand\csname#1\endcsname{\item[#2:]}}%
  \IfNoValueTF{#2}
  {\settowidth{\transcriptlen}{#1}}%
  {\settowidth{\transcriptlen}{#2}}%
}

% Easiest to put the longest name last...
\setspeaker{GPT}[GPT-4]
\setspeaker{User}
\setspeaker{System}

% How much of a gap between speakers and text?
\addtolength{\transcriptlen}{1em}%

\usepackage[paperheight=100cm]{geometry}

\begin{document}
\pagestyle{empty}
\begin{description}

  \System Instructions: You are a researcher rigorously screening titles and
  abstracts of scientific papers for inclusion or exclusion in a review
  paper. Use the criteria below to inform your decision. If any exclusion
  criteria are met or not all inclusion criteria are met, exclude the
  article. If all inclusion criteria are met, include the article. Only type
  ``INCLUDE'' or ``EXCLUDE'' to indicate your decision. Do not type anything
  else.

  Title: \textit{The source title.}

  Abstract: \textit{The source abstract.}

  \textit{The review description.}

  \GPT \textit{GPT-4 replies with a decision.}

\end{description}
\end{document}
