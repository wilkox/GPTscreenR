\documentclass{article}
\usepackage[T1]{fontenc}
\usepackage{xparse}
\usepackage{enumitem}
\setlist[description]{
  font={\sffamily\bfseries},
  labelsep=0pt,
  labelwidth=\transcriptlen,
  leftmargin=\transcriptlen,
}

\newlength{\transcriptlen}

\NewDocumentCommand {\setspeaker} { mo } {%
  \IfNoValueTF{#2}
  {\expandafter\newcommand\csname#1\endcsname{\item[#1:]}}%
  {\expandafter\newcommand\csname#1\endcsname{\item[#2:]}}%
  \IfNoValueTF{#2}
  {\settowidth{\transcriptlen}{#1}}%
  {\settowidth{\transcriptlen}{#2}}%
}

% Easiest to put the longest name last...
\setspeaker{GPT}[GPT-4]
\setspeaker{User}
\setspeaker{System}

% How much of a gap between speakers and text?
\addtolength{\transcriptlen}{1em}%

\usepackage[paperheight=100cm]{geometry}

\begin{document}
\pagestyle{empty}
\begin{description}

    \System You are helping academic researchers perform a scoping review. Your
    task is to screen a single source against the review criteria. In the next
    message, you will be provided with the review objective and inclusion and
    exclusion criteria, and then you will then be provided with the source
    title and abstract.

    \User \textit{The user-provided review description.}

    \System You must work step by step. FIRST, generate a numbered list of
    criteria that must be met for a source to be included.

    \GPT \textit{GPT-4 replies with a numbered list of criteria.}

    \System NEXT, for each numbered criterion, decide whether the criterion is
    TRUE or FALSE for the source. It is normal for the title and abstract to
    not have enough information to make a clear decision for every statement.
    For these situations, give your best guess. After giving your response of
    TRUE or FALSE, give a one sentence explanation for your response.

    \GPT \textit{GPT-4 replies with a decision and explanation for each criterion.}

    \System FINALLY, consider your decisions on whether the source meets the
    conclusion criteria. Respond with a single word, either INCLUDE or EXCLUDE,
    representing your recommendation on whether the source meets the inclusion
    criteria. Do not write anything other than INCLUDE or EXCLUDE.

    \GPT \textit{GPT-4 replies with its final recommendation.}

\end{description}
\end {document}
