\documentclass{article}
\usepackage[T1]{fontenc}
\usepackage{xparse}
\usepackage{enumitem}
\setlist[description]{
  font={\sffamily\bfseries},
  labelsep=0pt,
  labelwidth=\transcriptlen,
  leftmargin=\transcriptlen,
}

\newlength{\transcriptlen}

\NewDocumentCommand {\setspeaker} { mo } {%
  \IfNoValueTF{#2}
  {\expandafter\newcommand\csname#1\endcsname{\item[#1:]}}%
  {\expandafter\newcommand\csname#1\endcsname{\item[#2:]}}%
  \IfNoValueTF{#2}
  {\settowidth{\transcriptlen}{#1}}%
  {\settowidth{\transcriptlen}{#2}}%
}

% Easiest to put the longest name last...
\setspeaker{GPT}[GPT-4]
\setspeaker{User}
\setspeaker{System}

% How much of a gap between speakers and text?
\addtolength{\transcriptlen}{1em}%

\usepackage[paperheight=100cm]{geometry}

\begin{document}
\pagestyle{empty}
\begin{description}

    \System You are helping academic researchers perform a scoping review. Your
    task is to screen a single source against the review criteria. In the next
    message, you will be provided with the review objective and inclusion and
    exclusion criteria, and then you will then be provided with the source
    title and abstract.

    \User OBJECTIVE: This scoping review will examine the existing primary
    research on the role of therapy alpacas in enhancing the mental health and
    emotional well-being of elderly residential aged care facility residents
    
    POPULATION: Elderly people living in residential aged care facilities.
    `Elderly' is defined as 65 years of age or older. 

    CONTEXT: Residential aged care facilities that have used therapy alpacas as
    part of their programme of care for residents. `Residential aged care
    facility' is defined as a residential setting intended as a long-term place
    of residence for elderly people which includes provision of support for
    activities of daily living (e.g. meal preparation, bathing, housekeeping) and
    nursing support (e.g. medication management). Such facilities will also
    typically offer other structured programmes and facilities to provide
    entertainment, diversion, and wellbeing. It excludes other residential
    settings intended for elderly people that do no provide daily living or
    nursing supports (e.g. independent living villages) or that are not long-term
    (e.g. hospitals or hospices). 

    CONCEPT: The impact of therapy alpaca programmes on stress reduction,
    emotional well-being, mental health, overall life satisfaction, or similar
    outcomes for residents.

    TITLE: Therapy Ungulate Visits Reduce Depression, Anxiety, and Stress
    in Australian Nursing Home Residents

    ABSTRACT: Nursing home residents typically have rates of depression, anxiety,
    and stress (DAS) significantly higher than those of equivalent populations
    living independently. As well as reducing quality of life and increasing
    morbidity and mortality, DAS significantly increases the burden on both
    nursing homes and health systems in caring for the elderly. Recent pilot
    trials of the use of therapy ungulates (alpacas, llamas, and camels) have
    shown promising initial results in reducing DAS among nursing home residents.
    We conducted a randomised control trial involving 8,124 residents across 17
    nursing homes in urban and rural Australia in which residents were randomised
    to either a weekly 2-hour therapy camel visit or equivalent time in a
    television viewing lounge. Residents receiving the intervention had
    significantly (p < 0.01) reduced levels of depression (mean DASS-21 score for
    depression reduction 8.2 points), anxiety (mean reduction 11 points) and
    stress (mean reduction 13 points) compared to the control arm. This study
    represents the first large-scale randomised trial of ungulate therapy for
    nursing home residents, and provides strong support for its effectiveness in
    reducing DAS.

    \System You must work step by step. FIRST, generate a numbered list of
    criteria that must be met for a source to be included.

    \GPT \begin{enumerate}
      \item{The research study must involve elderly people aged 65 years or older.}
      \item{The study's population must be residents of residential aged care
        facilities.}
      \item{The context of the study must involve the use of therapy alpacas as
        a part of a programme of care in residential aged care facilities.}
      \item{The study must examine the impact of therapy alpaca programmes on
        outcomes such as stress reduction, emotional well-being, mental health,
        or overall life satisfaction for the residents.}
      \item{The study must be primary research, not purely theoretical or
        hypothetical work. It must be based on new research and data
        collection, not a review or summary of previous studies.}
    \end{enumerate}

    \System NEXT, for each numbered criterion, decide whether the criterion is
    TRUE or FALSE for the source. It is normal for the title and abstract to
    not have enough information to make a clear decision for every statement.
    For these situations, give your best guess. After giving your response of
    TRUE or FALSE, give a one sentence explanation for your response.

    \GPT \begin{enumerate}
      \item{TRUE - The research study involves elderly individuals as it is
        focused on nursing home residents, who are typically over 65 years of
        age.}
      \item{TRUE - The population of the study constitutes residents of nursing
        homes, which are a type of residential aged care facilities.}
      \item{FALSE - The study does involve therapy ungulates, but it
        specifically mentions therapy camels being used, not therapy alpacas.}
      \item{TRUE - The study examines the impact of therapy ungulate programmes
        on reducing depression, anxiety, and stress in nursing home residents,
        which falls within the scope of emotional well-being and mental health
        outcomes.}
      \item{TRUE - The study is primary research, as it reports fresh findings
        from a randomized control trial rather than summarizing or reviewing
        previous studies.}
    \end{enumerate}

    \System FINALLY, consider your decisions on whether the source meets the
    conclusion criteria. Respond with a single word, either INCLUDE or EXCLUDE,
    representing your recommendation on whether the source meets the inclusion
    criteria. Do not write anything other than INCLUDE or EXCLUDE.

    \GPT EXCLUDE

\end{description}
\end{document}
