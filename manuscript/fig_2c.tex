\documentclass{article}
\usepackage[T1]{fontenc}
\usepackage{xparse}
\usepackage{enumitem}
\setlist[description]{
  font={\sffamily\bfseries},
  labelsep=0pt,
  labelwidth=\transcriptlen,
  leftmargin=\transcriptlen,
}

\newlength{\transcriptlen}

\NewDocumentCommand {\setspeaker} { mo } {%
  \IfNoValueTF{#2}
  {\expandafter\newcommand\csname#1\endcsname{\item[#1:]}}%
  {\expandafter\newcommand\csname#1\endcsname{\item[#2:]}}%
  \IfNoValueTF{#2}
  {\settowidth{\transcriptlen}{#1}}%
  {\settowidth{\transcriptlen}{#2}}%
}

% Easiest to put the longest name last...
\setspeaker{GPT}[GPT-4]
\setspeaker{User}
\setspeaker{System}

% How much of a gap between speakers and text?
\addtolength{\transcriptlen}{1em}%

\usepackage[paperheight=100cm]{geometry}

\begin{document}
\pagestyle{empty}
\begin{description}

    \System You are being used to help researchers perform a scoping review.
    You are not interacting directly with a user.

    A scoping review is a type of systematic review used to map the published
    scholarship on a topic. To gather relevant sources for a scoping review,
    the researchers search bibliographic databases for sources that might be
    relevant to the review, often using the Population, Concept, and Context
    framework. The titles and abstracts of sources that are found in this
    search are then screened against the review inclusion and exclusion
    criteria.

    Your task is to screen a single source against the study criteria. In the
    next message, you will be provided with the review objective and inclusion
    and exclusion criteria, and then you will then be provided with the source
    title and abstract.

    To screen the source, you must work step by step. First, generate a
    numbered list of statements that summarise the inclusion and exclusion
    criteria for the scoping review, including the Population, Concept, and
    Context if they are provided. The statements should be clear, comprehensive
    and complete. Any source for which all the statements are true is a source
    that meets the inclusion criteria. As a template, here are some example
    statements (these are a generic set of examples that are not related to the
    current scoping review):

    \begin{enumerate}[label=\arabic*.\ ]
    \item The source reports the results of a randomised control trial
    \item The source reports the results of a study in which:
      \begin{enumerate}[label=\arabic{enumi}\alph*.\ ]
      \item The participants were all male; AND
      \item The participants were all aged between 18 and 74 inclusive
      \end{enumerate}
    \item The source reports the results of a study conducted in the European
          Union.
    \end{enumerate}

    Aspects of the inclusion criteria with multiple elements should be broken
    down into separate points where possible. For example, instead of:

    \begin{enumerate}[label=\arabic*.\ ]
      \item The source reports on a study of men who live in the European
            Union.
    \end{enumerate}

    You should instead say:

    \begin{enumerate}[label=\arabic*.\ ]
    \item The source reports on a study of people who are:
      \begin{enumerate}[label=\arabic{enumi}\alph*.\ ]
        \item Male; and
        \item Living in the European Union.
      \end{enumerate}
    \end{enumerate}

    Then, continue to work step by step. Refer back to the set of statements
    you developed summarising the inclusion criteria. For each statement,
    decide whether or not the statement is true for the source described by the
    title and abstract. You must select from the following permitted responses:
    TRUE, FALSE, LIKELY TRUE, LIKELY FALSE, or NOT APPLICABLE. No other
    response is permitted. It is normal for the title and abstract to not have
    enough information to make a clear decision for every statement. There is a
    natural and normal amount of ambiguity in this process. For these
    situations, give your best guess, making use of your general knowledge, and
    deciding LIKELY TRUE or LIKELY FALSE. Responses like UNCLEAR or NOT ENOUGH
    INFORMATION are not permitted. After giving your response, give a one
    sentence explanation for your response. For example:

    \begin{enumerate}[label=\arabic*.\ ]
      \item TRUE. The abstract describes the study design as a randomised control
        trial.
      \begin{enumerate}[label=\arabic{enumi}\alph*.\ ]
        \item TRUE. The abstract mentions that all the participants were male.
        \item TRUE. The abstract mentions that all the participants were aged
        between 30 and 40.
      \end{enumerate}
      \item LIKELY TRUE. While the abstract does not explicitly state that the study
      was conducted in the European Union, it does mention that the participants
      were all employees in French and German factories, so it is likely that
      they all live and work in the European Union.
    \end{enumerate}

    Finally, consider your decisions on whether the title and abstract meet the
    conclusion criteria. Overall, is it likely true that the source meets the
    inclusion criteria? End your response with a single word on a new line, either
    INCLUDE or EXCLUDE, representing your recommendation on whether the source is
    likely to meet the inclusion criteria. The response must end with a line
    containing only one of these two words; any other reply will cause the
    automatic parsing of your response to fail, which will be troublesome for the
    user. The response must not end with a blank line."

  \User OBJECTIVE: This scoping review will examine the existing primary
  research on the role of therapy alpacas in enhancing the mental health and
  emotional well-being of elderly residential aged care facility residents
  
  POPULATION: Elderly people living in residential aged care facilities.
  `Elderly' is defined as 65 years of age or older. 

  CONTEXT: Residential aged care facilities that have used therapy alpacas as
  part of their programme of care for residents. `Residential aged care
  facility' is defined as a residential setting intended as a long-term place
  of residence for elderly people which includes provision of support for
  activities of daily living (e.g. meal preparation, bathing, housekeeping) and
  nursing support (e.g. medication management). Such facilities will also
  typically offer other structured programmes and facilities to provide
  entertainment, diversion, and wellbeing. It excludes other residential
  settings intended for elderly people that do no provide daily living or
  nursing supports (e.g. independent living villages) or that are not long-term
  (e.g. hospitals or hospices). 

  CONCEPT: The impact of therapy alpaca programmes on stress reduction,
  emotional well-being, mental health, overall life satisfaction, or similar
  outcomes for residents.

  TITLE: Therapy Ungulate Visits Reduce Depression, Anxiety, and Stress
  in Australian Nursing Home Residents

  ABSTRACT: Nursing home residents typically have rates of depression, anxiety,
  and stress (DAS) significantly higher than those of equivalent populations
  living independently. As well as reducing quality of life and increasing
  morbidity and mortality, DAS significantly increases the burden on both
  nursing homes and health systems in caring for the elderly. Recent pilot
  trials of the use of therapy ungulates (alpacas, llamas, and camels) have
  shown promising initial results in reducing DAS among nursing home residents.
  We conducted a randomised control trial involving 8,124 residents across 17
  nursing homes in urban and rural Australia in which residents were randomised
  to either a weekly 2-hour therapy camel visit or equivalent time in a
  television viewing lounge. Residents receiving the intervention had
  significantly (p < 0.01) reduced levels of depression (mean DASS-21 score for
  depression reduction 8.2 points), anxiety (mean reduction 11 points) and
  stress (mean reduction 13 points) compared to the control arm. This study
  represents the first large-scale randomised trial of ungulate therapy for
  nursing home residents, and provides strong support for its effectiveness in
  reducing DAS.

  \GPT Inclusion and exclusion criteria:

  \begin{enumerate}[label=\arabic*.\ ]
    \item The source reports on a study of people who are:
    \begin{enumerate}[label=\arabic{enumi}\alph*.\ ]
      \item 65 years old or older; and
      \item Living in residential aged care facilities; and
      \item Not living in settings intended for elderly people that do not
        provide daily living or nursing supports or that are not long-term.
    \end{enumerate}
    \item The source reports on a study conducted in Residential aged care
      facilities that have used therapy alpacas as a part of their programme of
      care for residents.
    \item The source reports on the impact of therapy alpacas programmes on
      stress reduction, emotional well-being, mental health, overall life
      satisfaction, or similar outcomes for residents.
  \end{enumerate}

  Screening of the source:

  \begin{enumerate}[label=\arabic*.\ ]
    \item 
    \begin{enumerate}[label=\arabic{enumi}\alph*.\ ]
      \item LIKELY TRUE. The ages of the participants are not mentioned
        specifically, but since they are referred to as "nursing home
        residents", it can be assumed that most are likely 65 or older.
      \item TRUE. The participants were all living in nursing homes.
      \item LIKELY TRUE. The study was conducted in nursing homes, which
        generally include provision of support for activities of daily living
        and nursing support, making them likely to fit the definition given.
    \end{enumerate}
  \item FALSE. While the utilizing therapy animals, the animals used in this
    study were camels, not alpacas.
  \item TRUE. The study reported on the impact of the therapy involving camels
    on depression, anxiety, and stress levels in nursing home residents.
  \end{enumerate}

  RECOMMENDATION:
  EXCLUDE

\end{description}
\end{document}
